% $Id$
% Author:	Daniel Bosk <daniel.bosk@miun.se>
\chapter{Teori}
\label{ch:theory}
\noindent
Rapportens teoristudie, ibland kallat bakgrundsmaterial, ska innehålla fakta 
som krävs för läsarens förståelse för den fortsatta rapporten.
Du sammanfattar här vad som tidigare är skrivet inom ditt område, till exempel
i uppslagsverk, vetenskapliga artiklar, kurslitteratur, tidskrifter, 
examensarbeten, dokument på webben, tekniska rapporter och standarder.
Förklara pedagogiskt med konkreta exempel och många illustrationer.
Skriv på en nivå så att någon med liknande utbildning som du kan förstå texten.

Visa att du har kännedom om sammanhanget och bakgrunden till ditt arbete, och 
inte bara om det arbete du själv har genomfört.
Förklara gärna syftet med den teknik du beskriver, och inte bara hur tekniken 
fungerar.
På D-nivå ska du visa att du har kännedom om forskningsfronten inom området, 
för att säkerställa att ditt arbete har ett visst nyhetsvärde.
Men gå inte för långt ifrån ditt forskningsproblem.
Ditt uppdrag är inte att skriva en lärobok som innehåller sådant som kan läsas 
på annat håll.
Det är viktigt att hitta en lämplig balans mellan bakgrundsmaterial och dina 
egna resultat.

Rubriken kan gärna vara ett ämne, till exempel \emph{GSM-standarden} eller 
\emph{Forskningsläget inom område X}.

Om din metod är att genomföra en kritisk litteraturstudie behövs normalt inte 
ett separat kapitel med bakgrundsmaterial, utan referaten av källorna 
sammanställs då i resultatkapitlet.
Din kritik av källorna och dina argument för en egen uppfattning placeras 
i slutsatskapitlet.


\section{Definitioner}
\label{sec:definitions}
\noindent
Termer och förkortningar som är viktiga för läsarens förståelse av den 
fortsatta framställningen förklaras i detta kapitel.
Första gången du i den löpande texten använder ett begrepp eller en förkortning 
ska du förklara det, även om det dessutom finns definierat i ett 
terminologiavsnitt.
När begrepp introduceras skrivs de med emfas.

Första gången en \emph{förkortning} (förk.) används skrivs den inom parentes 
efter dess förklaring, såsom exemplifieras i denna mening.

Använd svenska termer så långt det är möjligt.
Se Svenska datatermgruppens \citep{Datatermgruppen} rekommendationer på URL
\begin{center}
	\url{http://www.datatermgruppen.se/}.
\end{center}


\section{Att referera eller citera}
\label{sec:ref}
\noindent
Du refererar när du sammanfattar eller återger en text med egna ord.
Exempelvis:
Forsslund förespråkar mer berättande rubriker i tekniska rapporter och menar 
att man särskilt i underrubrikerna kan ge viktig information 
\citep{Forsslund1969raf}.

Du citerar när du ordagrant återger en fras, en mening eller ett stycke.
I normalfallet refererar man istället för att citera källor.
Du kan använda direkta citat om du har speciella skäl, till exempel om du vill 
återge vedertagna definitioner av begrepp, när du tycker att en författare 
formulerat sig på ett särskilt träffande sätt, när du behöver stöd av en 
auktoritet, eller när du vill visa att en författare har fel.

Korta citat omges med citationstecken.
Att citera Strömqvist kan vara en passande illustration i detta sammanhang:
''Det må vara svårt att skriva, men det är roligt också.'' \citep[sidan 
X]{Stromquist2000s}.

Långa citat kan återges i form av blockcitat.
Textmassan placeras då på sidan utan citationstecken, men med indrag, det vill 
säga något förskjutet åt höger, och med mindre teckenstorlek.
Källan anges i direkt anslutning till citatet.
\begin{quote}
	Det här är ett blockcitat vilket innebär indragning, mindre teckenstorlek, 
	rak vänstermarginal, inte nödvändigtvis rak högermarginal, och inga 
	citationstecken.
	Blockcitatet kan tillämpas vid mer än ungefär 50 ord.  Blockcitat avslutas 
	alltid med källhänvisning. \citep[sidan X]{Stromquist2000s}
\end{quote}

\subsection{Testing}
\noindent
\dots
Se \prettyref{sec:aim} \dots
Se också \prettyref{ch:introduction} \dots


\section{Källhänvisningar och -förteckning}
\label{sec:references}
\noindent
Att kopiera in en text utan att ange dess källa betraktas som plagiat och 
därmed allvarligt fusk.

En källförteckning (lista över referenser) upprättas i slutet av rapporten för 
att ge läsaren en samlad upplysning om samtliga källor som du refererar, 
citerar eller av annat skäl hänvisar till i den löpande texten.
Källor ska anges så noggrant att läsaren ska kunna kontrollera dem, om de finns 
tillgängliga via bibliotek eller på internet.
Det förekommer även att muntliga källor och annan korrespondens inkluderas 
i källförteckningen, men det är ovanligt i tekniska rapporter.

Använd vederhäftiga källor, gärna författade av auktoriteter på området.
Privata hemsidor och studentuppsatser har låg tillförlitlighet som källor, 
i synnerhet om studentuppsatsen har lägre nivå (A, B, C eller D) än det egna 
arbetet.
Var källkritisk, särskilt mot kommersiella försäljningsargument.

Ta endast med källor i förteckningen som du refererar eller citerar i den 
löpande texten, detta sker automatiskt i \LaTeX och Bib\TeX.
Samtliga källor som tas upp i källförteckningen ska vara kopplade till 
rapporten genom hänvisning i den löpande texten, enligt Vancouver-systemet, som 
är vanligt förekommande i rapporter i tekniska ämnen.

Enligt Vancouver-systemet ordnas källförteckningen i den ordning källorna 
återges i den löpande texten, och källhänvisningen anges i texten med en siffra 
inom hakparenteser, som i detta dokument.
% De anges även i denna ordning i källförteckningen.
Exempel på källhänvisning:
Enligt \citet{Eriksson2001dsf} kan dynamiska SFN ge betydande prestandavinster.
På liknande sätt refereras till källor på internet, exempelvis kan du se 
\citet{Wikipedia2010h264} för att läsa om \emph{H.264}.

Eftersom information på webben kan revideras ofta, och eftersom webblänkar kan 
upphöra att fungera, måste datum anges då du själv hämtade information från 
webbsidan.
Vid webbaserade källor krävs ibland anvisningar för hur källan kan hittas. Tänk 
på att kvaliteten på materialet på internet varierar.


\section{Illustrationer}
\label{sec:illustrations}
\noindent
Samtliga illustrationer (bilder, figurer, diagram, tabeller) i rapporten ska 
vara numrerade och försedda med en kort figur- eller tabelltext.
Därtill ska i anslutning till texten anges källhänvisning varifrån 
illustrationen är hämtad, om den inte är av egen produktion.
Se \prettyref{fig:automata} för ett exempel.

\begin{figure}
	\centering
	\includegraphics[width=7cm]{automata.eps} % for latex(1)
	% \includegraphics[width=7cm]{automata.pdf} % for pdflatex(1)
	\caption[Ett exempel på en automat.]{Ett exempel på en automat 
	\citep{Wikipedia2012at}.}
	\label{fig:automata}
\end{figure}

Samtliga illustrationer ska vara kopplade till rapporten genom hänvisning i den 
löpande texten.
Hänvisningarna skrivs på svenska med begynnande gemen, likt den ovan.
På svenska skrivs figur alltid med inledande gemen då de förekommer i löpande 
text.

\begin{table}
	\centering\small
  \begin{tabular}{r|cccccccccc}
    \hline\hline
    \(\alpha\) & a & b & c & d & e & f & g & h & i & j \\
    \(P_E(\alpha)\) & 8.2  & 1.5 & 2.8 & 4.3 & 12.7 & 2.2 & 2.0 &
    6.1 & 7.0 & 0.2 \\
    \(P_S(\alpha)\) & 9.3  & 1.3 & 1.3 & 4.5 & 9.9 & 2.0 & 3.3 &
    2.1 & 5.1 & 0.7 \\
    \hline\hline
    \(\alpha\) & k & l & m & n & o & p & q & r & s & t \\
    \(P_E(\alpha)\) & 0.8 & 4.0 & 2.4 & 6.7 & 7.5 & 1.9 & 0.1 & 6.0 & 6.3 &
    9.1 \\
    %\(P_S(\alpha)\) & 3.5 & 8.8 & 4.1 & 1.7 & 0.007 & 8.3 & 6.3 & 8.7 &
    \(P_S(\alpha)\) & 3.2 & 5.2 & 3.5 & 8.8 & 4.1 & 1.7 & 0.0 & 8.3 & 6.3 &
    8.7 \\
    \hline\hline
    \(\alpha\) & u & v & w & x & y & z & å & ä & ö \\
    \(P_E(\alpha)\) & 2.8 & 1.0 & 2.4 & 0.2 & 2.0 & 0.1 & 0.0 & 0.0 & 0.0 \\
    \(P_S(\alpha)\) & 1.8 & 2.4 & 0.03 & 0.1 & 0.6 & 0.02 & 1.6 & 2.1 &
    1.5 \\
    \hline\hline
  \end{tabular}
	\caption[Tabell av sannolikhetsfunktionen för bokstäver i det engelska 
	respektive svenska språket.]
	{Tabell av sannolikhetsfunktionen för bokstäver i det engelska och det 
	svenska språket, \(P_E\) respektive \(P_S\), angiven i procent med en 
	decimals noggrannhet \citep{Wikipedia2011lf}.}
  \label{tbl:freq}
\end{table}

Se \prettyref{tbl:freq} för ett exempel på en tabell.
Automatisk numrering fungerar på samma sätt som för figurer, titeln kommer dock 
automatiskt att vara tabell istället för figur.
Att Referera till tabeller i texten följer samma regler som för figurer och 
utförs på samma sätt.


\section{Matematiska formler}
\label{sec:maths}
\noindent
Ett exempel på hur matematik bör formuleras i skrift ges i följande meningar.
Låt \(a\) och \(n\) vara heltal sådana att \(\gcd(a,n) = 1\), det vill säga att 
\(a\) och \(m\) är relativt prima.
Då har vi att
\begin{equation}
	\label{eq:fermat-euler}
	a^{\varphi(n)} \equiv 1 \pmod n,
\end{equation}
där \(\varphi\) är Eulers \(\varphi\)-funktion.
Resultatet i \prettyref{eq:fermat-euler} är känt som Fermat-Eulers sats.

Ytterligare ett exempel:
Den effekt \(P_{i,j}\) som överförs från en basstation \(i\) till mobiltelefon 
\(j\) modelleras enligt
\begin{equation}
	\label{eq:power}
	P_{i,j} = C\cdot \frac{P_i G_{i,j}}{d^\alpha_{i,j}},
\end{equation}
där \(P_i\) är basstationens sändareffekt; \(d_{i,j}\) är avståndet mellan 
basstationen och mobiltelefonen; \(\alpha\) är en utbredningsexponent som är 
\(2\) i fri rymd och cirka \(3\) till \(4\) i stadsbebyggelse; \(C\) är en 
faktor som beror av antennförstärkning, kanalfrekvens och antennhöjd; samt 
\(G_{i,j}\) är en stokastisk variabel som återspeglar fädningens inverkan.
Utifrån \prettyref{eq:power} inses att den mottagna effekten är proportionerlig 
mot sändareffekten.

Ibland kan en ekvation eller härledning behövas delas upp på flera rader, likt 
ekvationerna \eqref{eq:basecase} och \eqref{eq:recursion}.
\begin{quotation}
	When the message-blocks \(M_i\) are processed, it is done accordingly:
	\begin{eqnarray}
		\label{eq:basecase}
		H_0 &=& G, \\
		\label{eq:recursion}
		H_{j+1} &=& E(H_j, T_j, M_j),
	\end{eqnarray}
	where \(T_j\) is the tweak-value with the bit-fields set correctly for the
	\(j\)th block. When all blocks are processed, the final \(H\) is the result 
	of the UBI chaining mode.
	This means that each block is processed uniquely, they all depend on each 
	other and the original message has now been compressed to a single 
	block-sized result.
\end{quotation}

Det är dock inte alltid nödvändigt att numrera ekvationerna, exempelvis om vi 
bara nämner summan mellan \(1\) och \(100\),
\begin{equation}
	\nonumber
	\sum^{100}_{i=1} i,
\end{equation}
utan att senare referera till den behövs ingen numrering.
