\title{%
  Good commits - Exercises
}
\author{Amanda Berg and Daniel Bosk}
\institute{%
  KTH EECS
}

\mode<article>{\maketitle}
\mode<presentation>{%
  \begin{frame}
    \maketitle
  \end{frame}
}

\mode*


\section{Exercise 1 - Commit messages}
\begin{frame}
\begin{block}{Count files}
    We wish to count the number of files in a directory and its sub-directories. We have the following piece of code:
\end{block}
\begin{block}{Code}
    \lstinline{find . | wc -l}
\end{block}
\end{frame}

\begin{frame}
\begin{block}{Code}
    \lstinline{find . | wc -l}
\end{block}

\begin{alertblock}{Note:}
    This code has a problem. Let's demonstrate!
\end{alertblock}
\end{frame}


\begin{frame}
\begin{block}{Old}
    \lstinline{find . | wc -l}
\end{block}

\begin{block}{New}
    We change it to \\
    \lstinline{find . -type f | wc -l}
\end{block}
\end{frame}

\begin{frame}
\begin{question}
What would be a good commit message for this change? Why? %fbf-question
\end{question}    
\end{frame}

\begin{frame}
\begin{block}{Examples}
    We will now look at three examples and discuss what is good and what is bad in them
\end{block}
\end{frame}

\subsection{Examples}
\begin{frame}
\begin{block}{Message 1}
    \enquote{Fixed bug}
\end{block}

\begin{question}
\begin{itemize}
    \pause
    \item What is good about this message?
    \pause
    \item What is bad about it?
\end{itemize}
\end{question}
    
\end{frame}

\begin{frame}
\begin{block}{Message 2}
    \enquote{Fix bug where all elements, both files and folders, are counted each time the code is run}
\end{block}

\begin{question}
\begin{itemize}
    \pause
    \item What is good about this message?
    \pause
    \item What is bad about it?
\end{itemize}
\end{question}
    
\end{frame}


\begin{frame}
\begin{block}{Message 3}
    \enquote{Count files only\\ \vspace{10 pt} 
    This commit fixes a bug where all files and folders were counted instead of just the files}
\end{block}

\begin{question}
\begin{itemize}
    \pause
    \item What is good about this message?
    \pause
    \item What is bad about it?
\end{itemize}
\end{question}
    
\end{frame}

% Make the students write a new (improved) commit message (fbf)

\section{Exercise 2 - Commit plan}

\begin{frame}
  \begin{block}{Golden rule of version management}
    \begin{itemize}
      \item Single-purpose commits
      \item Small commits
      \item Only commit working code
    \end{itemize}
  \end{block}
\end{frame}

\begin{frame}
  \begin{exercise}
    \begin{itemize}
      \item You should plan the commits for writing three small scripts:
        \begin{enumerate}
          \item A script that counts the number of files in the current 
            directory and its subdirectories.
          \item A script that counts the number of \emph{(sub)directories} in 
            the current directory and its subdirectories.
          \item A script that:
            \begin{itemize}
              \item creates a new directory and initializes it as a Git repo.
              \item creates a base for a LaTeX report in there and commits it.
            \end{itemize}
        \end{enumerate}
      \item When would you commit and what would you include?
    \end{itemize}
  \end{exercise}
\end{frame}

\begin{frame}
  \begin{solution}
    \begin{itemize}
      \item In essence, we can have one commit per point.
        \begin{enumerate}
          \item A script that counts the number of files in the current 
            directory and its subdirectories.
          \item A script that counts the number of \emph{(sub)directories} in 
            the current directory and its subdirectories.
          \item A script that:
            \begin{itemize}
              \item creates a new directory and initializes it as a Git 
                repo.
              \item creates a base for a LaTeX report in there and commits 
                it.
            \end{itemize}
        \end{enumerate}
    \end{itemize}
  \end{solution}
\end{frame}

\begin{frame}
  \begin{exercise}
    \begin{itemize}
      \item In the same case:
        \begin{enumerate}
          \item A script that counts the number of files in the current 
            directory and its subdirectories.
          \item A script that counts the number of \emph{(sub)directories} in 
            the current directory and its subdirectories.
          \item A script that:
            \begin{itemize}
              \item creates a new directory and initializes it as a Git repo.
              \item creates a base for a LaTeX report in there and commits it.
            \end{itemize}
        \end{enumerate}
      \item When would it be suitable with a branch?
    \end{itemize}
  \end{exercise}
\end{frame}

\begin{frame}
  \begin{solution}
    \begin{itemize}
      \item Script \ref{TwoStepScript} is suitable to make a branch for.
        \begin{enumerate}
          \item A script that counts the number of files in the current 
            directory and its subdirectories.
          \item A script that counts the number of \emph{(sub)directories} in 
            the current directory and its subdirectories.
          \item \label{TwoStepScript} A script that:
            \begin{itemize}
              \item creates a new directory and initializes it as a Git 
                repo.
              \item creates a base for a LaTeX report in there and commits 
                it.
            \end{itemize}
        \end{enumerate}
      \item Why?
        Because it takes more than one commit to complete.
    \end{itemize}
  \end{solution}
\end{frame}

