% $Id$
Föreläsningen introducerar grundläggande LaTeX, det som behövs för akademisk 
rapportskrivning.
LaTeX är ett system för typsättning av dokument, det vill säga att dokumentet 
framställs på ett sätt som drar nytta av kunskaperna från århundraden av 
boktryckning -- hur texten blir så behaglig och effektiv som möjligt att läsa.
(Detta till skillnad från vanliga ordbehandlare, se \citetitle{memdesign} 
\cite{memdesign} för en diskussion om detta.)
Andra fördelar är att det är enkelt att systematisera och fokus hamnar enbart 
på innehållet, LaTeX sköter hur innehållet presenteras, numrera korrekt, 
hantera referenser, etc.

Inför föreläsningen bör du har läst kapitlen 1--3 
i \citetitle{Oetiker2011lshort} \cite{Oetiker2011lshort}.
